% !TEX TS-program = xelatex
% !TEX encoding = UTF-8

% This is a simple template for a XeLaTeX document using the "article" class,
% with the fontspec package to easily select fonts.

\documentclass[11pt]{article} % use larger type; default would be 10pt

\usepackage{fontspec} % Font selection for XeLaTeX; see fontspec.pdf for documentation
\defaultfontfeatures{Mapping=tex-text} % to support TeX conventions like ``---''
\usepackage{xunicode} % Unicode support for LaTeX character names (accents, European chars, etc)
\usepackage{xltxtra} % Extra customizations for XeLaTeX
\usepackage{multicol}

%\setsansfont{Deja Vu Sans}
%\setmonofont{Deja Vu Mono}

% other LaTeX packages.....
\usepackage{geometry} % See geometry.pdf to learn the layout options. There are lots.
\geometry{a4paper} % or letterpaper (US) or a5paper or....
%\usepackage[parfill]{parskip} % Activate to begin paragraphs with an empty line rather than an indent

\usepackage{graphicx} % support the \includegraphics command and options

\title{Mechanical Vibrations of Spring Systems}
\author{Matthew Burke and Ananth Mohan}
%\date{} % Activate to display a given date or no date (if empty),
         % otherwise the current date is printed 

\begin{document}
\maketitle

\begin{multicols}{2}

\begin{flushleft}
\section{Introduction}

The goal of this project is to demonstrate the mathematical and visual properties of mechanical vibrations. These vibrations are visualized through a mass-spring system, where a hanging solid has its position, velocity and acceleration determined by a set of environmental variables. Mechanical vibrations are described with a set of differential equations: when solved, we can find the position equation $y(t)$. This solution was used to graph the position of an object using JavaScript in a web browser application, where the use can set the environmental variables.

\end{flushleft}

\section {}

\end{multicols}




\end{document}